\documentclass[12pt,a4paper]{article}
\usepackage{amsmath,amscd,amsbsy,amssymb,latexsym,url,bm,amsthm}
\usepackage{epsfig,graphicx,subfigure}
\usepackage{enumitem,balance}
\usepackage{wrapfig}
\usepackage{mathrsfs,euscript}
\usepackage[usenames]{xcolor}
\usepackage{hyperref}
\usepackage[vlined,ruled,linesnumbered]{algorithm2e}
\usepackage{float}
\hypersetup{colorlinks=true,linkcolor=black}

\newtheorem{theorem}{Theorem}
\newtheorem{lemma}[theorem]{Lemma}
\newtheorem{proposition}[theorem]{Proposition}
\newtheorem{corollary}[theorem]{Corollary}
\newtheorem{exercise}{Exercise}
\newtheorem*{solution}{Solution}
\newtheorem{definition}{Definition}
\theoremstyle{definition}

\renewcommand{\thefootnote}{\fnsymbol{footnote}}

\newcommand{\postscript}[2]
 {\setlength{\epsfxsize}{#2\hsize}
  \centerline{\epsfbox{#1}}}

\renewcommand{\baselinestretch}{1.0}

\setlength{\oddsidemargin}{-0.365in}
\setlength{\evensidemargin}{-0.365in}
\setlength{\topmargin}{-0.3in}
\setlength{\headheight}{0in}
\setlength{\headsep}{0in}
\setlength{\textheight}{10.1in}
\setlength{\textwidth}{7in}
\makeatletter \renewenvironment{proof}[1][Proof] {\par\pushQED{\qed}\normalfont\topsep6\p@\@plus6\p@\relax\trivlist\item[\hskip\labelsep\bfseries#1\@addpunct{.}]\ignorespaces}{\popQED\endtrivlist\@endpefalse} \makeatother
\makeatletter
\renewenvironment{solution}[1][Solution] {\par\pushQED{\qed}\normalfont\topsep6\p@\@plus6\p@\relax\trivlist\item[\hskip\labelsep\bfseries#1\@addpunct{.}]\ignorespaces}{\popQED\endtrivlist\@endpefalse} \makeatother

\begin{document}

\noindent

%========================================================================
\noindent\framebox[\linewidth]{\shortstack[c]{
\Large{\textbf{Lab04-Matroid}}\vspace{1mm}\\
CS214-Algorithm and Complexity, Xiaofeng Gao, Spring 2021.}}
\begin{center}
\footnotesize{\color{red}$*$ If there is any problem, please contact TA Haolin Zhou.}

% Please write down your name, student id and email.
\footnotesize{\color{blue}$*$ Name:Beichen Yu  \quad Student ID:519030910245 \quad Email: polarisybc@sjtu.edu.cn}
\end{center}

\begin{enumerate}
\item \textit{Property of Matroid.} 
\begin{enumerate}
	\item
	Consider an arbitrary undirected graph $ G=(V,E) $. Let us define $ M_{G}=(S,C) $ where $ S=E $ and $ C=\left\{I \subseteq E \mid\left(V, E \backslash I\right) \text { is connected}\right\} $. Prove that $ M_{G} $ is a \textbf{matroid}.\par
	    \begin{proof}
	    	(Hereditary) If removing $I$ does not disconnect the graph $G$, then removing any subset of $I$ will not disconnect $G$ either. So it holds hereditary.\\
	    	
	    	(Exchange property) For convenience, let $n=|V|$. Let $A \in C$, which means $(V,E \backslash A)$ is a connected graph. So $|E\backslash A| \geqslant n-1$. Consider a set $B$ that $|B|<|A|$, then we have $|E\backslash B| \geqslant |E\backslash A|$. So $|E\backslash B| \geqslant n$, which means there is at least a circle in $(V,E \backslash B)$. \\
	    	
	    	Because $|E\backslash B| \geqslant |E\backslash A|$ and both of them is connected, we can make sure that the number of edges in all circles in $(V,E \backslash B)$ is bigger than that in $(V,E \backslash A)$. So we can prove that we can find a edge $e$ in a circle in $(V,E \backslash B)$ that is not used in $(V,E \backslash A)$. \\
	    	
	    	If it is false, then the edges in circles in $(V,E \backslash B)$ is less or equal then that in $(V,E \backslash A)$, however with same number of vertices, the edges out of circles in $(V,E \backslash B)$ can not be more then that in $(V,E \backslash A)$.\\
	    	
	    	 That means $e \in A$ while $e \notin B$, which is $e \in A \backslash B$. As $e$ is in a circle, we can remove it from $(V,E \backslash B)$ without disconnecting the graph. So $B \cup {e} \in C$.  That means exchange property is held.\\
	    	
	    	As it holds both hereditary and exchange property, it is indeed a matroid.
	    	
	    	
	    \end{proof}

	\item
	Given a set $A$ containing $n$ real numbers, and you are allowed to choose $k$ numbers from $A$. The bigger the sum of the chosen numbers is, the better. What is your algorithm to choose? Prove its correctness using \textbf{matroid}.\par
	    \begin{solution}
	    	Obviously we should sort the $n$ real numbers in descending order at first, and then choose the first $k$ numbers. The sum of the chosen numbers is the biggest.\\
	    	
	        Denote $\mathbf{C}$ be the collection of all subsets of $A$ that contains no more than $k$ elements. Now we will try to prove that  $(A,\mathbf{C})$ is a matroid.
	        
	        (Hereditary) If $I \in \mathbf{C}$, then $I$ contains no more then $k$ elements. Obviously any subset of $I$ contains no more then $k$ elements, so $(A,\mathbf{C})$ is hereditary.
	        
	        (Exchange property) Denote $A \in \mathbf{C}, B \in \mathbf{C}$ and $|A| < |B|$. We can find a element $x$ that $x \in A$ and $x \notin B$ because $|A| < |B|$. According to the definition we have $|A|<|B|<k$, so $|A| < k-1$. So $|A \cup {x}| < k, A\cup {x} \in \mathbf{C}$. That means $(A,\mathbf{C})$ is exchange property.
	        
	        As $(A,\mathbf{C})$ holds both hereditary and exchange property, it is indeed a matroid.\\
	        
	        Next we prove that our solution is just equal to Greedy-MAX algorithm.
	        
	        Find the smallest element $s$ in the $A$. And next we denote the weighted function $c:A \rightarrow \mathbb{R}^+$. For each element $x \in A$, we denote $c(x) = x - s + 1$. Then the weight function extend to subset of $A$ by summation:
	        $$c(I) = \sum_{x\in I}c(x)$$
	        Then we can using the Greedy-MAX algorithm. First sort all elements in descending order of $c(x)$, and add elements to the solution set $S$ constantly until the solution set $S \notin  \textbf{C}$, which means there is already $k$ elements in $S$. Because the descending order of $c(x)$ is the same as the descending order of $x$, so our solution is the same as the Greedy-MAX algorithm.
	        
	     
	    \end{solution}

\end{enumerate}
\item \textit{Unit-time Task Scheduling Problem.} Consider the instance of the \textbf{Unit-time Task Scheduling Problem} given in class. 
    \begin{enumerate}
        \item Each penalty $\omega_{i}$ is replaced by $80-\omega_{i}$. The modified instance is given in Tab.~\ref{tab:1}. Give the final schedule and the optimal penalty of the new instance using Greedy-MAX.
		\begin{table}[H]
			\setlength{\abovecaptionskip}{0.cm}
			\setlength{\belowcaptionskip}{0.5cm}
			\centering
			\caption{Task}
			\label{tab:1}			
			\begin{tabular}{|c|ccccccc|}
				\hline
				$ a_{i} $&1&2&3&4&5&6&7\\
				\hline
				$ d_{i} $&4&2&4&3&1&4&6\\
                \hline
                $ \omega_{i} $&10&20&30&40&50&60&70\\
				\hline
			\end{tabular}
		\end{table}
	        \begin{solution}
	            The Greedy-MAX selects $a_1,a_2,a_3,a_4$, then rejects $a_5, a_6$, and finally accepts $a_7$. So the final schedule is $\langle a_2,a_4,a_1,a_3,a_7,a_5,a_6\rangle$. The final optimal penalty is $80 - \omega_5 + 80 - \omega_6 = 80 - 50 + 80 - 60 = 50$.
	        \end{solution}
        \item Show how to determine in time $O(|A|)$ whether or not a given set $A$ of tasks is independent. (\textbf{Hint}: You can use the lemma of equivalence given in class)
 	        \begin{solution}
                According to the lemma of equivalence given in class, the set $A$ is independent means for $t = 0,1,2,\cdots,n,N_t(A) \leqslant t$. We can assume that there are $n$ elements in set $A$. \\First of all, we should use an array $a[t]$ to record the number of tasks whose deadline is $t$. This step takes $O(n)$ to finish. \\Then we should calculate the partial sum of $a[t]$ and record it in $b[t]$. The value recorded in $b[t]$ is in fact the value of $N_t(A)$. It takes another $O(n)$. \\Finally we need to check for $t = 0,1,2,\cdots,n$, whether $b[t] \leqslant t$, and it takes $O(n)$ as well. \\So the time complexity is equal to $O(n+n+n) = O(n) = O(|A|)$.
            \end{solution}
    \end{enumerate}

\item \textit{MAX-3DM.} Let $X$, $Y$, $Z$ be three sets. We say two triples $\left(x_{1}, y_{1}, z_{1}\right)$ and $\left(x_{2}, y_{2}, z_{2}\right)$ in $X \times Y \times Z$ are \textit{disjoint} if $x_{1} \neq x_{2}$, $y_{1} \neq y_{2},$ and $z_{1} \neq z_{2}$. Consider the following problem:
    
    \begin{definition}[MAX-3DM] 
        Given three disjoint sets $X$, $Y$, $Z$ and a non-negative weight function $c(\cdot)$ on all triples in $X \times Y \times Z$, \textbf{Maximum 3-Dimensional Matching} (MAX-3DM) is to find a collection $\mathcal{F}$ of disjoint triples with maximum total weight.
    \end{definition}

    \begin{enumerate}
    	\item Let $D = X \times Y \times Z$. Define independent sets for MAX-3DM.
    	\item Write a greedy algorithm based on Greedy-MAX in the form of \emph{pseudo code}. \label{Item-Greedy}
    	\item Give a counter-example to show that your Greedy-MAX algorithm in Q.~\ref{Item-Greedy} is not optimal.
    	\item Show that: $\max\limits_{F \subseteq D} \frac{v(F)}{u(F)} \leq 3$. {\color{blue}(Hint: you may need Theorem~\ref{Thm-Intersect} for this subquestion.)} 
    	\end{enumerate}
    	    \begin{solution}
    	        \begin{enumerate}
    	        	\item A set $I\subset D$ is an independent set if and only if for any two triples in $I$ are disjoint.
    	        	\item 
    	        	\begin{minipage}[t]{0.8\textwidth}
        \begin{algorithm}[H]
            \BlankLine
            \caption{Greedy-MAX}
            Sort all the triples in $D$ by weight function $c(\cdot)$ so that $c(d_1) \geqslant c(d_2) \geqslant ... \geqslant c(d_m)$;

            $S \leftarrow \varnothing$;

            \For{$i = 1$ \textbf{to} $m$} {
                \If{for any triple $d$ in $S$, $d$ and $d_i$ are disjoint} {
                    $S \leftarrow S \cup \{d_i\}$
                }
            }
            \Return{$S$};
        \end{algorithm}
        \end{minipage}
        
        \item We set $A=\{1,2\}, B=\{3,4\} ,C = \{5,6\}$, and $c(2,3,5) = 4, c(1,3,5) = 3, c(2,4,6) = 2$, and for any other triple $d\in D$, $c(d) = 0$. If using the Greedy-MAX algorithm, we will choose $c(2,3,5) = 4$ and $c(1,4,6) = 0$ with total weight is equal to 4. However if we choose $c(1,3,5) = 3, c(2,4,6) = 2$, and the total weight is 5.
        \item For $i \in \{1,2,3\} $, we denote that a set ${F_i}$ is an i-th independent set if and only if for any two triples in ${F_i}$, the the numbers in the i-th dimension are not equal. We denote $\mathcal{I}_{i}$ as a collect of all i-th independent set. Then we try to prove that $M_i = (D,\mathcal{I}_{i})$ is a matroid. \\
        
         (Hereditary) A set $A \in \mathcal{I}_{i}$ means $\forall d$ in $A$, the numbers in the i-th dimension are not equal. So for any subset of $A$, the numbers in the i-th dimension are not equal as well. So $M_i$ satisfies hereditary.\\
         
         (Exchange property) Suppose $A,B \in \mathcal{I}_{i}$ and $|A| < |B|$. We can make sure that there must be a triple $d$ in $B$ that is different with any triple in $A$ in the i-th dimension, otherwise we will find $|A| \leqslant |B|$. Then we have $A \cap {d} \in \mathcal{I}_i$ because in the i-th dimension $d$ is different with any triple in $A$. Then $M_i$ satisfies hereditary.\\
         
         So  $M_i = (D,\mathcal{I}_{i})$ is indeed a matroid.\\
         
         If we the independent set we defined in (a) $I$, then we get an independent system $(D, \mathcal{I})$. Obviously $\mathcal{I} = \mathcal{I}_{1} \cup \mathcal{I}_{2} \cup \mathcal{I}_{3}$. Now according to \textbf{Theorem~\ref{Thm-Intersect}}, we have $\max\limits_{F \subseteq D} \frac{v(F)}{u(F)} \leq 3$.
         
    	        \end{enumerate}
    	    \end{solution}
    
    \begin{theorem} \label{Thm-Intersect}
        Suppose an independent system $(E, \mathcal{I})$ is the intersection of $k$ matroids $\left(E, \mathcal{I}_{i}\right)$, $1 \leq i \leq k$; that is, $\mathcal{I}=\bigcap_{i=1}^{k} \mathcal{I}_{i}$. Then $\max\limits_{F \subseteq E} \frac{v(F)}{u(F)} \leq k$, where $v(F)$ is the maximum size of independent subset in $F$ and $u(F)$ is the minimum size of maximal independent subset in $F$.
    \end{theorem}    
\end{enumerate}

\vspace{20pt}

\textbf{Remark:} You need to include your .pdf and .tex files in your uploaded .rar or .zip file.

%========================================================================
\end{document}
